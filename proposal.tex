\documentclass[12pt]{IEEEtran}
\usepackage[utf8]{inputenc}
\usepackage[english]{babel}
\usepackage{csquotes}
\usepackage{biblatex}

\addbibresource{references.bib}

\title{Applications and Implementation of Differential Privacy on Statistical
Databases}
\author{Jonathan Sumner Evans, Victoria Girkins, and Sam Sartor}

\begin{document}

\maketitle

% (1) an abstract
\begin{abstract}
Statistical databases have long been know to be vulnerable to inference attacks.
One method of mitigating these attacks is Differential Privacy. Our final
project will implement a working Differential Privacy statistical database and
then perform experiments using the database to analyze how well our
implementation achieves the objectives of Differential Privacy databases.
\end{abstract}

% (2) introduction (including the background, motivation, and goal)
\section{Introduction}
Statistical databases are designed for statistical analysis of the datasets
contained within the database. One of the desired properties of statistical
databases is that nothing about an individual should be learnable from the
database that cannot be learned without access to the
database~\cite{Dwork:2006:DP}. In other words, gathering information about the
records underlying the statistics using allowed queries on the database should
be impossible.

Without any protections, well crafted queries to a statistical database can
reveal information about a single row. An attack of this form is called an
\textit{inference attack}. Take, for example, a class of 30 people that stores
the students' homestate and age.  If it is known from some other source that
only one of them is from a certain state, determining the age of that person is
easy and merely requires a calculation of the \texttt{sum} of the ages of people
with that homestate. In this case, since there is only one person, the sum will
be equal to that person's age.

One way to alleviate this problem and reduce the risk of a successful inference
attack is to use a technique known as \textit{Differential Privacy} --- a term
first coined by Cynthia Dwork at Microsoft Research in
2006~\cite{Hilton:DP:history}. The goal of Differential Privacy is to maximize
the statistical accuracy of data while also maximizing the privacy of the
individual records in the database. It does this by introducing randomness into
the dataset while still maintaining the statistical accuracy of the queries on
that dataset. This randomness reduces the likelihood of obtaining information
about the actual individual records in the database.

% (3) key idea and approach (e.g., survey, testbed, system
% design/implementation, experiments, etc.)
\section{Approach}
Differential Privacy has received adoption in many applications including the
United States Census Bureau to show commuting
patterns~\cite{Machanavajjhala:2008}. Apple also advertises its use of
Differential Privacy on their Privacy page~\cite{Apple:2017}. Because of this,
our team has decided to take the previous research on the topic and attempt to
implement a Differential Privacy statistical database.

Our team will research the best approach for implementing this, but currently,
we are considering two approaches:
\begin{enumerate}
    \item Taking an existing statistical database and adding a Differential
        Privacy component; or
    \item Creating a simple statistical database with a Differential Privacy
        component built in.
\end{enumerate}

Initially, we will limit the number of statistical functions that we implement
to the minimum required to determine whether or not our implementation functions
as intended. If time allows, we will add more advanced statistical functions to
our implementation.

% (4) expected results
\section{Expected Results}
\label{expected-res}
After implementing this database, our team will run experiments to determine
whether or not our implementation achieves the following objectives:
\begin{enumerate}
    \item The individual records are reasonably difficult to determine.
    \item The statistical data returned by the queries from the database are
        sufficiently accurate.
\end{enumerate}
If these two objectives are met, then the project will be considered a success.

% (5) conclusion
\section{Conclusion}
Statistical databases are useful tools for allowing users to analyze the
datasets contained within the database while maintaining the privacy of
individual records. There are challenges with this including the possibility of
inference attacks but these can be mitigated by using techniques such as
Differential Privacy. Our team's objective is to create a Differential Privacy
statistical database and perform experiments to determine if our implementation
achieves the objectives described in Section~\ref{expected-res}.

% (6) references
\printbibliography

\end{document}
