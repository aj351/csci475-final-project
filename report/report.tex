\documentclass[conference,11pt]{IEEEtran}
\usepackage[T1]{fontenc}
\usepackage[utf8]{inputenc}
\usepackage[english]{babel}
\usepackage{csquotes}
\usepackage{biblatex}
\usepackage{amsmath}

\addbibresource{../references.bib}

\title{Applications and Implementation of Differential Privacy on Statistical
Databases}
\author{
    \IEEEauthorblockN{
        Jonathan Sumner Evans\IEEEauthorrefmark{1},
        Victoria Girkins\IEEEauthorrefmark{2} and
        Sam Sartor\IEEEauthorrefmark{3}
    }
    \IEEEauthorblockA{
        Department of Computer Science,
        Colorado School of Mines\\
        Golden, Colorado\\
        Email:
            \IEEEauthorrefmark{1}jonathanevans@mines.edu,
            \IEEEauthorrefmark{1}vgirkins@mines.edu,
            \IEEEauthorrefmark{2}rdmerillat@mines.edu,
    }
}

\begin{document}

\maketitle

% (1) an abstract
\begin{abstract}
Statistical databases have long been know to be vulnerable to inference attacks.
One method of mitigating these attacks is Differential Privacy. For our final
project we implemented a simple Differential Privacy statistical database and
then performed experiments using the database to analyze how well our
implementation achieves the objectives of Differential Privacy databases.
\end{abstract}

% (2) introduction (including the background, motivation, and goal)
\section{Introduction}
\subsection{Previous Attempts at Anonymizing Statistical Data}
Many attempts have been made to privatize data in the past and many are still in
use today. However everything from simply removing columns containing personally
identifiable information to advanced techniques such as as k-anonymity and
l-diversity have been shown to be vulnerable to attack~\cite{Atockar:2014}.
K-anonymity, for example, does not include any randomization, attackers can
still make inferences about data sets~\cite{Aggarwal:2005}.

\subsection{Differential Privacy}

% (3) key idea and approach (e.g., survey, testbed, system
% design/implementation, experiments, etc.)
\section{Approach}

% (4) activities (such as literature study and experiments) that you performed
\section{Experiment}
\subsection{Dataset}
\subsection{Crafted Queries}
\subsection{Database Implementation}

% (5) description and analysis of the key results and observations
\section{Results}

% (6) discussion of the limitations and potential future work
\section{Limitations and Future Work}

% (7) conclusion
\section{Conclusion}

% (8) references
\printbibliography

\end{document}
